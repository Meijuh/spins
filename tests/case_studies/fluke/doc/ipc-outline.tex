% %%%%%%%%%%%%%%%%%%%%%%%%%%%%%%%
%
% CS 611 Fluke Verification Project
%
% Description of Fluke IPC
%
% %%%%%%%%%%%%%%%%%%%%%%%%%%%%%%%

% Document setup info
\documentstyle{article}

%%%%%%%
%%%%%%% Some standard \def's

\long\def\com#1{}
\long\def\xxx#1{{\em {\bf Fix: } #1}}

% We don't need no steenkin' equations - just gimme a working underscore!
\catcode`\_=\active


%%%%%%%
%%%%%%%

\topmargin 0pt
\advance \topmargin by -\headheight
\advance \topmargin by -\headsep
\textheight 8.9in
\oddsidemargin 0.3in
\evensidemargin \oddsidemargin
\marginparwidth 0.5in
\textwidth 6in

%% Heh.. Not sure what this does... but it allows me to include postscript(?)
%\input{psfig}

\title{\Large \bf IPC Support Specification}

\author{Ajay Chitturi~~~~John McCorquodale~~~~Patrick Tullmann~~~~\\
        Jeff Turner~~~~Godmar Back \\[2ex]
        {\tt veripc@jensen.cs.utah.edu}
        }

%BEGIN DOCUMENT--------------------
\begin{document}

\maketitle

%% XXX fill all this in!

% this was an open-ended comment that cut stuff off ...
% \com{

\subsection{IPC Setup}


\subsection{IPC machine dependent macros}

{\tt EXC_SET_IP(ip)} and {\tt EXC_GET_IP()} are wrappers that set the
machine depenedent instruction pointer.  They are used for setting a
new kernel entry point.  For example, during a {\tt
fluke_client_connect_send_over_receive()}, after the ``connect'' phase
is complete, the kernel will set the IP to be the {\tt
fluke_send_over_receive()} entry point.  To exit the kernel, 
the IP to {\tt fluke_nop__entry}.

{\tt IPC_FINISH_RECEIVE(thread, status)}.  This macro sets the
user-return code for a kernel syscall, and sets the thread's
IP to {\tt fluke_nop__entry}.  It also does some silly stuff
with the ``min_msg'' registers--but only because they aren't
real registers.

{\tt IPC_STATUS(thread)} sets the status code for the current ipc
syscall.  The details are quite grungy:  On the x86 the eax register
is overloaded.  The bottom 16 bits are the number of buffers
the invoker is using (an in/out parameter), the top 16 bits are
the return code of the IPC call.

{\tt EXC_RETURN_INSANITY(cond)} set the {\tt exc_state.code} to
the given insanity condintion, and returns a KR_USER_EXCEPTION.


\subsection{Ipc Pickle Operations}

\begin{itemize}

\item {\tt void ipc_pickle(s_thread_t *client, s_thread_t *server)}

Takes client and server threads.  They must be pointing to each other
respectively.  Moves the active pointers into indirect links.  Zero's
out the direct pointers.
 
\item {\tt void ipc_unpickle(s_thread_t *client, s_thread_t *server)}

Destroy the indirect pointers.  (We don't have to ``follow'' them,
because they were passed in as parameters.)

Check that the two threads agree on the ``direction'' of the
connection--

\xxx{What do the FLUKE_THREAD_/FOO/_SENDER bits mean?}

\xxx{Shouldn't the code assert() that the passive links are correct?}

The server's direct client pointer is set to the client and
vice-versa-vice.

\item {\tt int ipc_client_unpickle(s_thread_t *client, int no_wait)}

Client must be the current thread.  It must have no direct server
pointer.  (Its trying to restore the direct pointer.)

First, the indirect server link is turned into a direct pointer. If
the link is bogus, 0 is returned.  (The object is locked (the {\tt sob}
lock) if the link was successfully followed.)  If the link pointed to a
non-thread object, then unlock the object and return.

\xxx{Couldn't server be waiting for _its_ server to pickle itself?}

If the server is in WAIT_IPC_UNPICKLE, meaning that it is waiting
for its ``other half'' to pickle a connection,  then we unlock the
server.  The thread_wakeup() call put the thread in the WAIT_NONE
state, and its not on any queues, so we have it ``captured''.

If the captured server's {\em passive client link} isn't pointing at
us, then the client blocks in WAIT_IPC_PICKLE and hands off to the
server.  (An exception being, that if the no_wait flag is passed in
non-zero, then the server is readied and we return immediately without
blocking.) Otherwise, call {\tt ipc_unpickle(client, server)} which
will restore the active links.  Then ready the server, as we had it
captured.

But, if the server wasn't in WAIT_IPC_UNPICKLE, then if we're the
no_wait flag was passed in non-zero, 

\item {\tt int ipc_server_unpickle(s_thread_t *server, int no_wait)}

\end{itemize}




%%%% ----------------------------
\section{Support State \& Functions}

These definitions are {\em {\sc NOT}} precise enough to generate
formal definitions from.  They are here so that you can look up what a
function does without having to mentally parse C code.  The meanings
of various bits of state is also described.

\subsection{Wait States}

{\tt wait_state.val} is a flag describing the current state of
a thread that is blocked in the kernel.  Wait states are {\em both}
flags and masks.  They are ``ordered'' in that some wait states
are more specific versions of other wait states.  The most
``generic'' wait state is {\tt WAIT_ANYTHING}.  (No thread is
every actually put in this state, but a specific version of
this state.)  For example, a thread blocked in {\tt
WAIT_IPC_SRV_SENDER} is also blocked as {\tt WAIT_CANCELABLE} and
as {\tt WAIT_ANYTHING}.  The full list of states breaks down as
the following tree illustrates:

\begin{verbatim}
 * WAIT_NONE
 * WAIT_ANYTHING ->
 *      WAIT_IPC_PICKLE
 *      WAIT_STOPPED
 *      WAIT_CANCELABLE ->
 *              WAIT_ON_COND
 *              WAIT_IPC_SRV_DSCN
 *              WAIT_IPC_SRV_RECEIVER ->
 *                      WAIT_IPC_SRV_RECV
 *                      WAIT_IPC_SRV_ASEND
 *              WAIT_IPC_SRV_SENDER ->
 *                      WAIT_IPC_SRV_ORECV
 *                      WAIT_IPC_SRV_SEND
 *              WAIT_IPC_CLI_DSCN
 *              WAIT_IPC_CLI_RECEIVER ->
 *                      WAIT_IPC_CLI_RECV
 *                      WAIT_IPC_CLI_ASEND
 *              WAIT_IPC_CLI_SENDER ->
 *                      WAIT_IPC_CLI_ORECV
 *                      WAIT_IPC_CLI_SEND
 *              WAIT_IPC_UNPICKLE
 *              WAIT_IPC_IDEMPOTENT ->
 *                      WAIT_IPC_EXCEPTION
 *                      WAIT_IPC_CALL
 *                      WAIT_IPC_SECURITY
\end{verbatim}

All of the values defined satisfy the property that (NODE \& \verb\<\{}NODE of
any child\verb\>\ ) == NODE.  I.e., a bitwise AND of the wait state and itself
or any of its "child" states will return itself.
So, if you see {\tt thread_wakeup(thread, WAIT_IPC_SRV_SENDER)},
then it will return {\tt WAIT_IPC_SRV_SENDER} 
if the thread is in {\tt WAIT_IPC_SRV_SENDER},
{\tt WAIT_IPC_SRV_ORECV}, or {\tt WAIT_IPC_SRV_SEND}.

\subsection{IPC connection state}

Existing, usable IPC connections can be in two states: active, and
pickled.  Active connections are internally represented as a direct
pointer from on thread to another. (ie, {\tt client\verb\->\ipc_state.server}
is the address of the server's s_thread_t object).  Pickled connections
are represented as {\em links}.  One or the other is accurate, never
both (the non-active link is 0.)  A direct pointer can be
thought of as a 'cached' value of the link.

\subsection{thread_wakeup(target_thread, required_state)}

Disables interrupts, spin locks 'target_thread's wait_state lock
and then calls {\tt id_thread_wakeup_locked} which wakes up and
captures 'target_thread' iff its current wait value is
{\em at least} 'required_state'. (See the Wait State section for
details on the wait states.)

If 'target_thread' is waiting in {\tt WAIT_ON_COND} state, then it is
removed from the queue and returned to the WAIT_NONE state.  The old
wait state of the target thread is returned.
(Note that there is a potential race condition handled in here, too.
If a thread is woken from a condition variable wait, the kernel
can't change its wait state until after its been removed from the
wait queue.  Thus in this code, if the thread is waiting on a
condition variable, but is not actually on the specified list, then
it is in the process of being removed by {\tt id_cond_wakeup()} which
has removed it from the list, but has not updated its state.)

If 'target_thread' is not in {\tt WAIT_COND} then 'target_thread's
old wait state is returned and it is put into the WAIT_NONE state.

If 'target_thread' is not in at least 'required_state' then 0 is
returned.

\subsection{thread_handoff(current_thread, state, target_thread)}

\textbf{Executive Summary:} Give control to 'target_thread' atomically,
put ourselves in waiting 'state', and handle the possibility of
us getting canceled.

\textbf{Assertions:} 'current_thread' must be running and be different
from 'target_thread'. 'state' must be different from both
\texttt{WAIT_NONE} and \texttt{WAIT_ON_COND}.

Disable interrupts, acquire spinlocks on both wait states in defined
order to avoid deadlock, make sure that our own wait state is
\texttt{WAIT_NONE}. Then, put 'target_thread' on the ready queue
using \texttt{id_thread_ready} (because it's locked), release
the lock and let the other thread go. That's all that's done to
the other thread.

At this point, a cancel may be already pending (indicated by
\texttt{wait_state.cancel_pending}). If one is pending, then there are
two cases:
\begin{itemize}
    \item The state we want to wait in is {\em not cancelable} (i.e.,
        it doesn't have \texttt{WAIT_CANCELABLE} in its pattern - unlike
        most of the intermediate IPC states) - then set our resume_rc
        to \texttt{KR_CANCEL}. Note that we don't return here.
        This is a somewhat special case.
    \item If, however, the state we should wait in is {\em cancelable},
        then unlock the current thread object, enable interrupts again
        and return immediately with \texttt{KR_CANCEL}.
        Don't context switch. Consider yourself canceled.
\end{itemize}
At this point, resume_rc is either set to \texttt{KR_RESTART} or
to \texttt{KR_CANCEL}, depending on whether we have a cancelation
pending or not. We now set the current thread's wait state to 'state'
and go to the dispatcher.

Cryptic comment \#21a in the code says:
\begin{verbatim}
/*
 * We don't need to add the current thread to any wait queue,
 * because the wait state itself defines the "ownership" of the thread.
 */
\end{verbatim}
This simply means that another thread will explicitly point at us and
make us ready - rather than we being one thread among dozens waiting
for an event to happen.

Eventually, the dispatcher will reschedule us (because someone else
put us on the ready queue), and then we reenable interrupts and return
with either \texttt{KR_RESTART} or \texttt{KR_CANCEL}, depending on
whether a cancelation was pending or not when we invoked the
dispatcher.

Note that the ``unusual'' case, where the dispatcher is invoked
even though we already knew a cancelation was pending is only used
when actually handling a cancelation.
\xxx{Figure out where exactly this is used.}


\subsection{thread_wait(current_thread, state, other_thread, other_state)}

\textbf{Executive Summary:} Put current_thread to sleep in the given
state, (like thread_handoff) unless the other_thread is already in
other_state.  Perhaps perform an atomic action on other_thread before
we go to sleep.


\textbf{Assertions:}
\begin{itemize}
  \item 'current_thread' must be the actual ``current thread''
  \item 'current_thread' must be different than 'target_thread'
  \item 'state' must be neither \texttt{WAIT_NONE} nor \texttt{WAIT_ON_COND}
\end{itemize}

\textbf{Assumptions:}

None

\textbf{Locks manipulated}
\begin{itemize}
  \item current_thread\verb\->\wait_state.lock (spin)
  \item target_thread\verb\->\wait_state.lock (spin)
\end{itemize}

\textbf{Description:}

This function is similar to thread_handoff() above (80\% of the code is
identical), but instead of simply handing off to another thread, it
checks to see if other_thread has reached other_state.  If so,
thread_wait returns immediately with \texttt{KR_RESTART}.  Otherwise,
the current_thread is put to sleep in wait_state state.

The function first disables interrrupts, and grabs spin locks for both
of the threads' wait_states.

If other_state is equal to 0, the other_thread is canceled before
current_thread is put to sleep.  Also, if other_state is equal to -1,
it is a magic flag to unlock the other_thread\verb\->\sob.lock.  Both of these
actions must be performed atomically due to potential race conditions
with the other_thread waking up before current_thread is really asleep.

At this point, a cancel may be already pending (indicated by
\texttt{wait_state.cancel_pending}). If one is pending, then there are
two cases:
\begin{itemize}
    \item The state we want to wait in is {\em not cancelable} (i.e.,
        it doesn't have \texttt{WAIT_CANCELABLE} in its pattern - unlike
        most of the intermediate IPC states) - then set our resume_rc
        to \texttt{KR_CANCEL}. Note that we don't return here.
        This is a somewhat special case.
    \item If, however, the state we should wait in is {\em cancelable},
        then unlock the current thread object, enable interrupts again
        and return immediately with \texttt{KR_CANCEL}.
        Don't context switch. Consider yourself canceled.
\end{itemize}
At this point, resume_rc is either set to \texttt{KR_RESTART} or
to \texttt{KR_CANCEL}, depending on whether we have a cancelation
pending or not. We now set the current thread's wait state to 'state'
and go to the dispatcher.

Cryptic comment \#21a in the code says:
\begin{verbatim}
/*
 * We don't need to add the current thread to any wait queue,
 * because the wait state itself defines the "ownership" of the thread.
 */
\end{verbatim}
This simply means that another thread will explicitly point at us and
make us ready - rather than we being one thread among dozens waiting
for an event to happen.

Eventually, the dispatcher will reschedule us (because someone else
put us on the ready queue), and then we reenable interrupts and return
with either \texttt{KR_RESTART} or \texttt{KR_CANCEL}, depending on
whether a cancelation was pending or not when we invoked the
dispatcher.

Note that the ``unusual'' case, where the dispatcher is invoked
even though we already knew a cancelation was pending is only used
when actually handling a cancelation.
\xxx{Figure out where exactly this is used.}

\subsection{thread_cancel(thread)}

\textbf{Executive Summary:}  Set the cancel pending bit in thread.

\textbf{Assertions:}
\begin{itemize}
  \item Assert thread doesn't already have a cancel pending, OR
        that its in an uncancelable wait.
\end{itemize}

\textbf{Assumptions:}

\textbf{Locks manipulated}
\begin{itemize}
  \item spin locks wait_state.lock
  \item disables interrupts
\end{itemize}

\textbf{Description:}

Disables interrupts, locks the wait_state.lock, and sets
thread.cancel_pending. If it can wakeup the thread, it does
so, sets the thread's resume_rc to KR_CANCEL and readies it.

\subsection{Functions in ipc.c}
\begin{itemize}
\item ipc_client_disconnect(client_thread)\\ 
Make sure any connection we are involved in is actually active, by
unpickling the connection if it is pickled. (Since unpickling a
connection may involve waiting this means that the standard disconnect
is a {\it soft} disconnect, which means the other side can potentially
delay arbitrarily). Now wake up the server thread if it is waiting in
some wait state (i.e. its wait_state is WAIT_ANYTHING), nuke the
server corresponding to the client and the client corresponding to the
server and put the server on
the ready queue. Otherwise
cancel the server thread and put yourself to sleep in the wait state
{\tt WAIT_IPC_CLI_DSCN}. 

\item ipc_server_disconnect(server_thread)\\ 
Similar to ipc_client_disconnect (described above) except that 
the server needs to break any (XXX) idempotent connection as well.

\item ipc_client_find_server(client_thread, disconnect_code)\\ 
Ensures that the appropriate reliable connection exists and is
active. This involves unpickling a pickled connection. If the server
field is still zero after a successful return from the
ipc_client_unpickle() function check the disconnect_code parameter. If
it is FLUKE_IPC_RECV_DISCONNECTED do a IPC_FINISH_RECEIVE().
\xxx{explain IPC_FINISH_RECEIVE()}
Otherwise IPC Status is set to the disconnect_code and the IP of the
EXC state is set to fluke_nop__entry. Return KR_RESTART in either
case.
\xxx{List IP and status in the IPC state at the beginning and come up
  with some clean terminology}

\item ipc_server_find_client(server_thread, disconnect_code)\\ 
Similar to ipc_client_find_server() described above.

\xxx{Organize this stuff better}
\item Functions implementing components of reliable IPC paths\\ 
Most paths are built from two phases: 1(send) and 2(receive).
All phase 1 routines are called with the current thread in an unknown
state, and on successful return, the other thread is captured as the
receiver. All pahse 2 routines are called with the other thread
captured as receiver, and return with everything all done, ready to
return from the syscall. Routines that implement both phases 1 and 2
are marked _12_.

\item ipc_client_1_connect_send(client_thread, connected_ip)\\ 
  First disconnect any connection client_thread is involved
  in. Establish a
  connection to a new server thread using s_port_capture(). Set the
  server field of the client and the client field of the server
  appropriately and change the flags of the client and the server to
  show that the client is now a client_sender and the server can no
  longer be a server_sender. Transfer a min_msg(XXX) from the client
  to the server. Set the IP field of the client to the connected_ip
  parameter and the IP field of the server to
  fluke_ipc_server_receive__entry. Then perform a reliable transfer
  operation from the client to the server.  \xxx{semantics of the
    connected_ip parameter}

\item ipc_client_1_ack_send(client_thread, acked_ip)\\  
  First make sure the connection is active (by calling
  ipc_client_find_server). If the server is not waiting to send
  (i.e. it's wait state is not at least WAIT_IPC_SRV_SENDER) put the
  client thread to sleep in the WAIT_IPC_CLI_ASEND state and return. 
  If the server is waiting to send, capture the server. If the server
  wasn't waiting in WAIT_IPC_SRV_ORECV (i.e. it is not waiting for a
  over and receive) put the client thread to sleep in the wait state
  WAIT_IPC_CLI_ASEND, put the server on the ready queue (by calling
  thread_handoff) and return. If the above condition doesn't hold this
  means bot the client and the server are ready to reverse. Change the
  flags of the server and the client to indicate that the client will
  now be a sender and that the server is no longer a sender. Transfer
  min_msg(XXX) from the client to the server. Set the IP of the client
  to the acked_ip parameter and that of the server to
  fluke_ipc_server_receive__entry. Now perform a reliable transfer
  from the client to the server.
  
\item ipc_client_1_send(client_thread, out_wval)\\
  First make sure the connection is active (by calling
  ipc_client_find_server). If the server is not waiting as a receiver
  (i.e. it's wait state is not at least WAIT_IPC_SRV_RECEIVER) put the
  client_thread to sleep in the wait state WAIT_IPC_CLI_SEND and
  return. Otherwise capture the server as a receiver. If the server
  was waiting to receive (i.e. in the wait state 
  WAIT_IPC_SRV_RECV) do a reliable transfer from the client to the
  server. If the server was waiting in the wait state
  WAIT_IPC_SRV_ASEND, this means that the server is acking the client before
  it is done and so just throw away the send data.

\item ipc_client_2_over_receive(client_thread, wval)\\
  \xxx{wval could be just IPC_SRV_RECEIVER or IPC_SRV_ASEND.. What is
    it semantically}
  If the wval parameter is WAIT_IPC_SRV_RECV do a IPC_FINISH_RECEIVE
  \xxx{What does FINISH do}(Since the server still thinks it's
  receiving data notify it that the message has ended).
  Put the client to sleep in the wait state WAIT_IPC_CLI_ORECV and put
  the server on the ready queue.
  \xxx{how does release sound for the the opposite of capture instead
    of put the server on the ready queue}

\item ipc_client_12_receive(client_thread)\\
  First make sure the connection is active (by calling
  ipc_client_find_server). If the server is not waiting as a sender
  (i.e. it's wait state is not WAIT_IPC_SRV_SENDER), put the client
  thread to sleep in the wait state WAIT_IPC_CLI_RECV (by calling
  thread_wait) and return. Otherwise capture the server thread as the
  sender. If the the server is not waiting to send (i.e. it's wait
  state is not WAIT_IPC_SRV_SEND, it means that the server is done
  sending; so release the server, do a FINISH_RECEIVE(XXX) and return
  KR_RESTART. Otherwise put the client to sleep in the wait state
  WAIT_IPC_CLI_RECV and release the server.

 \xxx{What are the semantics of the sys_ipc_client_receive() function
 which seems to be the only one calling this function}\\
 \xxx{need to explain codes and ip parameters etc being passed to
   functions properly}\\


\textbf{Server-side reliable IPC path components}\\
\item ipc_server_1_ack_send(server_thread, acked_ip)\\
 Similar to ipc_client_1_ack_end() described above.

\item ipc_server_1_send(server_thread, out_wval)\\
 Similar to ipc_client_1_send() described above.

\item ipc_server_2_wait_receive(server_thread, wval)\\
 Break the connection with the client thread and release it. Nuke the
 client field of the server and the server field of the client. Do a
 FINISH_RECEIVE() (XXX), set the IP to fluke_ipc_wait_receive__entry
 and put the client on the ready queue and then 
 wait on a port set for an incoming IPC request.

\item ipc_server_2_over_receive(server_thread, wval)\\
 Similar to ipc_client_2_over_receive() described above.

\item ipc_server_3_wait_receive(server_thread)\\
 Set the IP of the server to fluke_ipc_wait_receive__entry and wait
 on a port set for an incoming IPC request.

\item ipc_server_12_receive(server_thread)\\
 Similar to ipc_client_12_receive() described above.

\end{itemize}

\subsection{Functions in {\tt sys_ipc.c}}
This section gives the flavor of IPC (one-way/idempotent/reliable)
corresponding to each of the functions in {\tt sys_ipc.c}.
Note that {\tt sys_ipc_setup()} is used in all three flavors, and
{\tt sys_ipc_wait_receive()} is used in one-way and reliable IPC.

\noindent \textbf{One-way IPC:}

\begin{itemize}
\item  sys_ipc_send()
\item  sys_ipc_wait_receive()
\item  sys_ipc_setup()
\end{itemize}

\noindent \textbf{Idempotent IPC:}

\begin{itemize}
\item sys_ipc_call()
\item sys_ipc_reply()
\item sys_ipc_setup()
\end{itemize}

\noindent \textbf{Reliable IPC:}

\begin{tabular}{ll}
 & \\
 {\bf Client}                               & {\bf Server}                           \\ \hline
 sys_ipc_client_connect_send()              & sys_ipc_wait_receive()                 \\
 sys_ipc_client_ack_send()                  & sys_ipc_reply_wait_receive()           \\
 sys_ipc_client_send()                      & sys_ipc_server_ack_send()              \\
 sys_ipc_client_connect_send_over_receive() & sys_ipc_server_send()                  \\
 sys_ipc_client_ack_send_over_receive()     & sys_ipc_server_ack_send_wait_receive() \\
 sys_ipc_client_send_over_receive()         & sys_ipc_server_ack_send_over_receive() \\
 sys_ipc_client_receive()                   & sys_ipc_server_send_wait_receive()     \\
 sys_ipc_client_disconnect()                & sys_ipc_server_send_over_receive()     \\
                                            & sys_ipc_server_receive()               \\
                                            & sys_ipc_server_disconnect()            \\
                                            & sys_ipc_setup()
\end{tabular}

\subsection{Kernel-internal exception codes}
This section describes kernel return codes used within the kernel
to flag exception conditions that require interrupting a kernel operation.

\begin{enumerate}

\item {\tt KR_MIN} Minimum kernel-internal return code value. any
value greater than or equal to this is a kernel-internal code.

\item {\tt KR_USER_EXCEPTION} A processor exception occured which
should be blamed on the user (e.g. because the exception was generated
while accessing user space). The current thread's exception_state
contains the details.

\item {\tt KR_PAGE_FAULT} Page fault occurred.
This gets turned into a real KR_USER_EXCEPTION by the kentry layer
if the page fault cannot be resolved in the kernel
and no appropriate region keeper can be found to handle the fault

\item {\tt KR_CANCEL} Another thread is trying to manipulate us and
has asynchronously canceled us,  e.g. due to thread_interrupt(),
thread_get_state(), thread_set_state().

\item {\tt KR_NO_MEMORY} Ran out of kernel memory

\item {\tt KR_RESTART} This return code indicates that we have context
switched due to a wait, and we need to restart execution in user mode
before doing anything else, since in a nonblocking
(stack-per-processor) kernel any variables still on the kernel stack
will still refer to the old thread.

\item {\tt KR_SECURITY_FAULT} Security fault occurred.
An upcall to the security policy server is required. (We don't need
this probably).

\end{enumerate}

\end{document}
