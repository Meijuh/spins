% %%%%%%%%%%%%%%%%%%%%%%%%%%%%%%%
%
% CS 611 Progress Report
%
% %%%%%%%%%%%%%%%%%%%%%%%%%%%%%%%

% Document setup info
\documentstyle{article}
     
%%%%%%%
%%%%%%% Some standard \def's 

% We don't need no steenkin' equations - just gimme a working underscore!
\catcode`\_=\active 

\long\def\com#1{}
\long\def\xxx#1{{\em {\bf Fix: } #1}}

%%%%%%%
%%%%%%%

\topmargin 0pt
\advance \topmargin by -\headheight
\advance \topmargin by -\headsep
\textheight 8.9in
\oddsidemargin 0.3in  
\evensidemargin \oddsidemargin
\marginparwidth 0.5in
\textwidth 6in

%% Heh.. Not sure what this does... but it allows me to include postscript(?)
%\input{psfig}

\title{\bf \Large Fluke IPC Verification Progress Report}

\author{Ajay Chitturi~~~~John McCorquodale~~~~Patrick Tullmann~~~~\\
	Jeff Turner~~~~Godmar Back \\[2ex]
	{\tt veripc@jensen.cs.utah.edu}
	}

\date{November 12, 1996}

%BEGIN DOCUMENT--------------------
\begin{document}

%HEADER
\maketitle

%%%%%%% 
\section*{Project Outline}

We proposed to model Fluke reliable Interprocess Communication (IPC)
and verify that under normal execution the threads involved
will not deadlock.  In addition, we will verify that any IPC can be canceled
at arbitrary points, without leaving the threads involved deadlocked
or lost.  We will also generate a descriptive formal model of the IPC
implementation in Fluke and the objects related to it.

\section*{Status}

\subsection*{Clarified Direction}

We have clarified the goals of the project.  We have defined
what we will model in Promela, and how.

We will simulate kernel threads, mutexes, wait queues, and condition
variables.  Threads both represent the flow of control through the
kernel code and are objects that encapsulate the state of the thread.
Mutexes are the kernel mechanism for mutual exclusion.  Wait queues
are an abstraction for creating lists of threads that are waiting for
an event.  All threads, the first thread, or a specific thread can be
awoken (and removed) from a wait queue.  Condition variables are an
abstraction on top of wait queues.

This is the full extent of low-level kernel objects that are utilized
in in the kernel IPC code path.  Other kernel supported objects are
either tangential or irrelevant.  We have decided to exclude port and
port set objects, they are used so that one thread may {\em find}
another thread to communicate with.  We are only concerned with the
communication itself;  there is still a great deal of setup that
goes on once two threads have a handle on each other, which will
probably simulate, but we will just give thread handles directly. 

To give a feel for how the kernel handles IPC operations, we will look
at the {\tt sys_ipc_client_send()} operation which sends some data to
the thread's current server.  It does not reverse the direction of the
half-duplex channel (so the thread may do multiple sends before
awakening the server.)

First, the client ``refreshes'' its handle to the server (in the
common case, this is not necessary.)  Then it ``captures'' the server
thread; if the server is in the required state (waiting to receive,)
the transfer of data is made.  Barring any page faults during the
transfer, the client then puts the server back into the 
waiting-to-receive state (so it can transfer more data.)  Note that
during this whole operation, the server thread was never actually awoken.

\subsection*{Current Project Status}

\subsubsection*{Completed Work}

We completed an analysis of kernel source, looking for which 
objects and abstractions are used in the IPC path, and trying
to gain an understanding of the issues and workings of the code.
We generated html cross-references of the source tree with some
freely available tools.  This greatly aided our understanding
of the interdependencies of the code.

We all have gained a solid understanding of the C code that
is relevant to the IPC path, and have implemented many of
the basic abstractions that the IPC code is implemented with.
We have detailed the requirements and some implementation details of
all of the low-level abstractions.  See the attached ``Simulation
Specification'' for more details.  The decision on modeling threads
is, we think, the most interesting.

We have documented many of the kernel internal support functions,
constants, and idiosyncracies.  See the attached ``IPC Support
Specification'' for more details.

\subsubsection*{Work in progress}

We are currently working on implementing and testing the low-level
abstractions.  Mutexes are complete and tested;  Wait queues need
some testing; condition variables are under way.  The thread objects
are well thought-out, but we haven't started implementation.

One of the core kernel files where much of the interesting
inter-thread communication is captured, {\tt wait.c} is being
``translated'' to Promela.  This is providing a wealth of information
and letting us see what issues we will face in translating the Fluke
kernel source.

All of our work is in the Flux project CVS tree.  You can look at
our source tree in {\tt ~tullmann/class/cs611/verify/}.   Or,
a ``cvs checkout verify'' will check the tree out of the Flux root CVS
tree.

\section*{Remaining Work}

\begin{enumerate}
\item Complete testing of low-level abstractions.  The thread state
and control objects must be implemented first, then the wait queues
can be implemented on top of that.  Condtion variables will come after
that.  These components should all be testable on their own.

\item Finalize design issues in Promela implementation.  Specificially,
handling exceptional return codes, temporary variables (naming
issues), conventions for naming accessor functions, etc.  There are a
lot of issues that arise only because of deficiencies in Promela.
Most striking is the lack of ``functions.''

\item Finalize design of user-level ``drivers'' for test cases.  We
have a rough idea of how we will generate test cases for the IPC
operations, this will need to be worked out.

\item Implement lots of Promela.  We have a lot of C code that needs
to be effectively translated into Promela.  A lot of work has gone
into supporting the same abstractions as the C code, so the
translation should be straightforward (baring the problems outlined in
the previous item.

\item Scale back, tweak, bug fix.  As more and more of the system
runs, we will find out what is feasible (in terms of memory usage,
etc.) and what is not.
\end{enumerate}

We have a lot of momentum now.  We all understand the basics of the
kernel code, and the thread interactions.  We have a good set of base
abstractions to build on, and we've thought of many of the issues
involved.  We think that by the end of the week we will have all of
the abstractions implemented and tested, and a good start on the IPC
control code path implemented.

\end{document}