\subsection{Condition Variable Model}

The C implementation of condition variables
makes use of POSIX-like condition
variables. In addition the interface provides a call which allows a
thread to place another (captured) thread asleep on a condition
variable. Unlike POSIX conditions variables, the associated mutex
is not re-acquired before returning from a wait operation.

A condition variable is modeled using a {\bf wait queue} of the
threads waiting on the condition variable. There is no extra state
required to be maintained as the wait queues implement all the
required features.

\com {
% Not any more, Pat ``fixed'' this.
One change had to be made in the implementation of the condWait()
function because of the way our wait queues were implemented. The
Promela implementation of waitQueueAdd() blocks where as the C
implementation does not. So we had to place some of the statements that
should actually have been executed after the waitQueueAdd(), before it
and place the whole sequence in an atomic block. This leads to a
slight difference in semantics whereby some of the interesting
interleavings are ignored.
}
