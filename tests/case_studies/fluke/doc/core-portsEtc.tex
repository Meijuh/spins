\subsection{Port and Port Set Models}

Ports are Fluke objects representing targets for IPC requests.
Ports might represent CORBA or MOM objects, for example,
or in Fluke implementations implementing full protection,
they may be used as secure capabilities.
A single server may maintain a number of ports,
each distinguishing a particular object, client,
or object/client pair.

Port sets are objects that act as a rendezvous point
between client threads attempting to perform IPC to a port
and server threads waiting for incoming IPC requests on one or more ports.
Multiple ports can be attached to a single port set,
and multiple server threads can wait for incoming requests
on a single port set at once.
This allows a single ``pool'' of service threads
to service incoming IPC requests from many ports
(i.e. requests associated with many ``objects'' the server exports).

Our model assumes that threads to connect
to for IPC have been ``found'' (the purpose of ports, port sets, and
references). Thus, models of ports or port sets were not
implemented in detail.  The only relevant interfaces used are 
{\tt s_pset_wait_receive()} by servers and {\tt s_port_capture()} by
clients.  The semantics of these operations have been distilled from
the C code.  

% This stuff below has to be made more clear
\com{
Specifically, a thread will block if its ``other half''
hasn't arrived yet.  It will check its cancel pending bit before
blocking.  It will block on a condition variable, in the WAIT_ON_COND
wait state.  When it is awoken it will, by default, return a
KR_RESTART code to its caller.  Additionally, the server thread {\em
always} blocks.  If it arrives and a client is waiting for it, then it
will awaken the client and then block.  In this way it is implied that
the client has the server ``captured'' when it returns from 
{\tt s_port_capture().}}

In our Promela implementation, we model these two functions to implement
the interface and make use of as little state as possible. The
rendezvous point is actually implemented as a bit in the server's
state. If the rendezvous bit is set, it means that either the client
or the server is waiting for a connection.
So when a client wants to connect to a server, it checks the
rendezvous bit to see if the server is waiting for a connection. If
the server is waiting, the server is captured and the connection
is made. If the server is not waiting for a connection, the client
sets the rendezvous 
bit and waits for the server to notice it. When a server is waiting
for an IPC  connection, it checks the rendezvous bit and if a client
is waiting, it makes the connection with the client. Otherwise the
server sets the rendezvous bit and waits for the client. In either
case the server blocks and is captured by the client which eventually
establishes a connection with this server.

% Do we need to say something about the non-cancelability of the wait 
% out here ?

It is interesting to note that the majority of the bugs in our model
came through this interface.  Since we strayed from the actual
implementation of the ports we missed subtle nuances like the server
always being blocked, etc. 

