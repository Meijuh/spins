% %%%%%%%%%%%%%%%%%%%%%%%%%%%%%%%
%
% Fluke verification project simulation specification.
%
% %%%%%%%%%%%%%%%%%%%%%%%%%%%%%%%

% Document setup info
\documentstyle{article}
     
%%%%%%%
%%%%%%% Some standard \def's 

% We don't need no steenkin' equations - just gimme a working underscore!
\catcode`\_=\active 

\long\def\com#1{}
\long\def\xxx#1{{\em {\bf Fix: } #1}}
%% classHeader args are {class name/number} {assignement name/number} {date}
\def\classHeader#1#2#3{{\large\noindent {\sc Patrick Tullmann} \hfill #1
\newline {\bf #2} \hfill #3\newline \hrule}}

%%%%%%%
%%%%%%%

\topmargin 0pt
\advance \topmargin by -\headheight
\advance \topmargin by -\headsep
\textheight 8.9in
\oddsidemargin 0.3in  
\evensidemargin \oddsidemargin
\marginparwidth 0.5in
\textwidth 6in

\title{{\Large \bf Simulation Specification}}

\author{Ajay Chitturi~~~~John McCorquodale~~~~Patrick Tullmann~~~~\\
        Jeff Turner~~~~Godmar Back \\[2ex]
        {\tt veripc@jensen.cs.utah.edu}
        }

%\date{Nov. 7, 1996}

%BEGIN DOCUMENT--------------------
\begin{document}
\maketitle

{\bf {\LARGE 
This Document is out of date.  The two-threads per 
proctype was discarded.  Much of the rest has changed, too.
It is only included because some info is up to date, and someday
this document might be brought back in synch.  This was our
initial design document.  The Code Is Always Right.
}}

This document describes the simulation of core abstractions in the
Fluke kernel.  We simulate threads, mutexes, wait queues, and
condition variables.

This should be quite specific.  Specific enough to implement from.

\section{Simulated Threads}

We will simulate Fluke threads with a pair of procs.  One proc will
encapsulate the ``control flow'' of a thread, the other will enacpsulate
the ``state'' of a thread.  They are refered to consistently throughout
all of our documentation as the {\bf thread control proc} and the 
{\bf thread state proc}. (No acronyms!)

The motiviating reason behind hiding the thread state in a proc is to
avoid huge global variables.  SPIN should be able to do more
optimizations with encapsulated local variables, than with arbitrary
global state.
Also, by forcing
all communication to go through channels, an implicit mutex is put
around many simple accesses.  

Since all threads have access to all other thread's state, an array of
channels will be global---one channel per thread state proc. 

To minimize false locking (ie, locking that is entirely an artifact of
the nature of this setup) we will keep all operations done on a
channel request to a minimum.  They will be on the order of ``lock the
{\tt wait_state} lock'' or ``set {\tt ipc_state.client == 0}''.

\subsection{Thread Control Proc}

The thread control proc is where all of the logic goes, it should look
a {\em lot} like the actual C source code.  (Due mostly to the magic
of the C pre-processor...)

\subsection{Thread State Proc}

The thread state proc will simply be a large {\tt do} {\tt od} block
that waits for any possible input, twiddles some local variables, and
replies, if necessary.

Here is a list of the messages a thread state proc can receive, and
what it does with such a message.

\begin{description}

\item[{\tt waitQueueAddSelf(queueId, replyChannel)}]  This is the
message a thread control proc sends to its associated thread state
proc.  It is the only way for a thread to get added to any wait queue. 

\item[{\tt waitQueueChange(oldQueueId, newQueueId)}]  This is sent
when a thread wants to move a different thread to another wait queue.

\item[{\tt waitQueueRemove(queueId)}]  Send this when a thread removes
another thread from a specific wait queue.  The target thread is then
said to be ``captured''--its not on any wait queue, and its not on the
ready queue.

\item[{\tt threadReady()}]  Send this when a thread is to be put on
the ``ready queue''

\item[{\tt lock(targetThread, lockName[, callerId])}] 
Lock the given lock in the target thread.
Blocks until lock is locked. 
See section \ref{Locks} for a description of the locks simulated.
callerId may be appended by the client-side macro implementing this function.
It will only be used for asserting ownership, etc, and is not
strictly necessary for ``correctness''.
On the client-side, this may be called as either {\tt spin_lock()}, or
{\tt lock()}--just to make translation easier.

\item[{\tt unlock(targetThread, lockName[, callerId])}]
Just like lock, but differnt in the right way.

\end{description}

This list is by no means complete.

\subsection{Miscellaneous Thread Details}

In the kernel, threads are identified by the address of their
kernel structure.  We will id threads by the index of the
global channel array for communicating with the control proc.
The macro {\tt currentThread()} will evaluate to the id.

%%%% ------------------

\section{Simulated Mutexes}

The kernel differentiates between spin locks, and regular mutexes for
efficiency reasons.  We provide both interfaces, but both are
implemented exactly the same.

\xxx{Lots more detailed required.  Just copy out of source comments}

It turns out there there is no easy way to put the mutexes in the
thread state proc.  We use a strict rendevous channel to communicate
with between state and control.  The state proc {\em cannot} accept a
mutex lock message unless the the mutex is unlocked (otherwise the
{\em state} proc would block, which is wrong.  We've rememdied this by
making the associated mutexes global.

\subsection{Specific Locks}
\label{Locks}

\begin{description}
\item[{\tt s_ob}] 
The s_ob lock corresponds to a lock over generic parts of the object
in question.  In the kernel {\tt s_thread_lock()} actually manipulates
this lock.

\item[{\tt ipc_state.lock}]
Lock of the ipc state portion of a thread.

\item[{\tt wait_state.lock}]
Lock of the wait state portion of a thread. 

\end{description} 

%%%% ------------------

\section{Simulated Wait Queues}

\subsection{Issues}

\begin{itemize}
\item A thread can only move itself from the run state to a wait
queue.  

\item Only ever add another thread to a wait_queue if its already on some
wait queue.  (implies what the thread is blocked on can change under
it.)

\item Must be able to wake up a specific thread on a queue.  (implies either
that a message is directed to the thread (state??, control??) or that
the queue structure can direct the message.)

\item Must be able to wake up the ``next'' thread on a queue. (implies message
is delivered to the queue which knows who is first; or directly to
``next'' thread which makes ``next next'' thread head of queue; or
is implicit in Promela (much like a mutex.))  ``next'' does not imply
a strict FIFO order--Fluke doesn't require it.  Must avoid starvation
of blocked threads, though.

\item Sleeping thread's control block must be stopped. (it should do a
wait_queue(foo, self) at some point...)  To me this implies that the
control block should message to the state block, and the state block
should take care of which list, etc.  BUT, BUT, BUT, the state proc
itself CANNOT block, then no one can manipulate its state.  It must
know that its control proc is blocked on it (so it can wake up the
control proc when necessary.)

\item A thread can be removed from a wait queue without being added to
any other queue (this is done a lot in the ipc, where a thread
is removed from the queue its on, its state is twiddled, and
then its put on the run queue.  This is called ``capturing'' a thread.
It is completely under the control of the thread that did the wait
queue remove operation.

\item There is no run queue in promela, we must fake one by controling
when a thread control proc blocks and when it wakes up.

\item The _only_ time the wakeup-arbitrary-thread syntax is used is
in condition variables.  The only condition variable relevant to
our simulation is the thread-local {\tt stop_cond}.  This greatly
simplifies things.
\end{itemize}

\subsection{Simulation details -- Structure of a Wait Queue}

A queue is just a bit vector, one bit per existant thread.  If a
thread's associated bit it 1, then the thread is on the queue.

When an arbitrary thread is to be awoken, the waker-uper just picks a
thread from the bit vector.  To prevent starvation, an integer records
the last thread awoken.  Each time a new thread is to be awoken, start
with the next thread, and look higher for a new thread to wake
up.

There is a global array of wait queues.  Wait queue identifiers are
passed around (simple integers--the index into the global array.)

\subsection{Simulation details -- Using the Wait Queue}

Here's how they work in terms of the messages sent around.

\begin{itemize}
\item {\bf Being put on a queue (from not being on one)}

Thread control gets to a {\tt wait_queue_add(queue, thread)}
statement.  There are two cases:

\begin{itemize}

  \item {\tt thread == current_thread()} Send a special message to
  state proc stating to add self to a queue. (see below) Block until a
  message is returned.

  \item {\tt thread != current_thread()} See {\bf Moving a thread
        to a different wait queue} below.
        
\end{itemize}

Case 2 is simple, and should work just fine.

Case 1 is not simple.  Thread control sends a message to its
corresponding thread state, asking to be put on a queue (state should
{\tt assert()} it is not already on a queue.)  Thread control proc
blocks until a response comes back from the thread state proc (thread
control proc sent a channel for the reply with its request.)  Thread
state proc records the queue being blocked on.  And adds itself to the
correct queue.

\item {\bf Moving a thread to a different wait queue}

Threads cannot move themself between wait queues (they're on a queue,
they can't do jack.)  To switch a thread to a different wait queue,
just send a message to its thread state proc.  The thread state proc
will remove itself from the old wait queue and put itself on the
new wait queue.  The target thread state proc should {\tt assert()} it
is on some wait queue.

NOTE: 'moving' a thread to the run queue isn't obvious in Promela, as
there is no run queue.  The thread state proc could notice that its
being asked to get on the run queue and then signal its thread control
proc.  See {\bf Putting a thread on the run queue} below for more
details.

\item {\bf Removing a thread from a queue}

If a specific thread is being awoken, go to the next paragraph.
Otherwise, to find a thread, loop through the bit vector until a
non-zero entry is found and use that.  (Start at the lastThreadAwoken,
and reset that field when you're done.)

To do this send a message to the corresponding thread control proc.
It will unset its bit in the associated wait queue, and reply to the
thread requesting it to get off the list.  The thread control proc
should note that it is 'captured'.

\item {\bf Putting a thread on the run queue}

Threads cannot put themselves on the run queue (see above.)  (Well,
there really isn't an explicit run queue, the thread control proc just
needs to get a message back and its off and running.)  So, when a
thread state proc gets a thread_ready() message, it should reply to
its thread control proc which should wake it up.

\end{itemize}

%%%% ------------------
\section{Simulated Condition Variables}

\subsection{Issues}
\xxx{Are there any non-trivial issues}

\subsection{Simulation details - Structure of a Condition Variable}

A Condition Variable is a {\bf wait queue} (refer previous section) of the
threads waiting on the condition variable. There is no extra state
required to be maintained as the wait queues implement all the
required features.


\subsection{Simulation details - Using the Condition Variable}

The interface for condition variables as defined in {\tt cond.h} is
implemented using the wait queue interface.

\xxx{Nothing new being done - should we cover the interface here}

%%%% ------------------
\section{Miscellaneous Simulations}

\subsection{Local Freaking Variables}

Whilst Promela will let you declare variable inside nested scopes (ie
within an arbitrary \verb\{,}\ pair.) it won't scope them, and names
can conflict.  Thus you can't include a macro more than once in the
same proctype contains a ``local'' definition.  This is, of course,
braindamaged.  Any reasonable language would support this trivially.
There is no reason other than laziness and lack of insight not to.

To hack around this obtuse and malformed limit, we propose the
follwing:  all macros within a given file will use prefixed
local variables (ie, in {\tt ipcPickle.pr}, all temporaries
will be prefixed with {\tt ipcPickle_}.)  All of these
will be defined in a macro that the user of the functions
must call.  So, {\tt ipcPickle.pr} will define a macro
{\tt DEFINE_IPC_PICKLE_VARS} that should be expanded at the 
top of the thread control proc.  In this way, we can do
some optimal-reuse of local variables, without killing ourselves.

\subsection{Links}

Links are modeled as a single bit.  They may be either {\it passive} 
or {\it active}.  What the link actually points to is kept in a
separate variable.  For example, in the IPC state there is a {\tt
serverLink} and a {\tt server} field.  The link indicates whether the
associated server index is active or passive.

\xxx{What exactly does this show us?}

\subsection{Interrupts}

{\tt enable_interrupts()} and {\tt disable_interrupts()} are used
throughout the kernel source.  We do not initially plan to include
them in our simulations, but we provide empty macros, {\tt
enableInterrupts()} and {\tt disableInterrupts()}, respectively, that
can be defined at a later time.

\subsection{Exceptional returns}

In Fluke all kernel functions return one of a small set of error codes
(the {\tt KR_{\em foo}} codes).  In Promela, we will be using macros
instead of functions.  As macros have no notion of a return value, we
will define a global ``return code'' ({\tt rc}) per thread.  Macros
will set this to indicate their return code.  All of the macros 
will be written in Pascal-ish single-entry, single-exit style.  
{\tt if-else} blocks will be used to implement this.

\end{document}
