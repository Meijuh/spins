\section{Fluke Core Models}
\label{core}

While evolving our current Promela model of Fluke
IPC we have developed a support infrastructure which includes Promela
implementations of many low-level Fluke objects used during Fluke IPC 
sessions. We call this support infrastructure the
{\it Core Models}.  These models model such Fluke entities as mutexes,
condition variables, wait queues, links, ports and port sets, and the
Fluke kernel entry layer (which embodies behavior intimately related to
the cancellability of IPC operations).

Implementation details and the design rationale of these models are
presented here.  Where appropriate, the implementation of these models is
quite faithful to the C-language source from which they are derived.  When
particular divergence from the C-code has occurred, we note the underlying
motivation and discuss the semantic similarities between the two
implementations.

Since our model is heavily based on the actual implementation, we have
a high measure of confidence that the results obtained from model
checking are in fact applicable to the real system.

\subsection{Translation Artifacts and the Sliding {\tt d_step} Technique}

Since Promela does not support language-level procedures or named blocks,
we were forced to rely on C preprocessor macros to give us the ability to
name code segments and provide a function-like decomposition of source
code.  Since these macros are not lexical scopes, and many instances of
the same macro can be instantiated in a single lexical scope (proctype),
these macros cannot contain variable declarations, labels or other symbol
declarations or duplicate symbol declaration will occur.  Thus, most macros
assume the presence of a local variable called {\tt rc} which
is used to hold the ``return value'' from the ``function'' that the macro
embodies.  This variable can be interrogated by the user of the macro.

The Core Models are designed to allow the use of a
technique we call the {\it sliding {\tt d_step}}.  When working with
a model as large as the Fluke IPC model, the problem of state-space
explosion is an omnipresent threat.  Since the Core Models are largely
self-contained and contain no external references, individual Core
Models (or groups of them) can be verified together, and then 
their ``public methods'' can be encased with the SPIN/Promela 
{\tt d_step} statement, which will avoid an overall increase in 
state-space. 
%% "can be kept at the edge of manageability" is too wordy... 
%% but I'll leave it in since I can't immediately come up with something better
The positioning of these {\tt d_steps} is quite flexible due to the
hierarchical and rigidly encapsulated structure of our models.  Thus,
problems can be kept at the edge of manageability, to verify the Fluke
implementation as rigorously as possible using the hardware available.

\subsection{Mutex Model}
\label{mutex-model}

{\it Mutexes} are a mechanism for
achieving mutual exclusion among a set of concurrently executing threads.  
A mutex is an object that can be in one of two states, {\it locked} and 
{\it unlocked}.  Mutex objects must support {\it lock} and {\it unlock},
which move the mutex from the unlocked state to the locked state, and
vice versa. Mutual exclusion is guaranteed since the lock and unlock
operations are atomic and only one contender will successfully acquire
the lock. Attempts to lock a mutex that is already locked by some 
other thread will cause the calling thread to wait until the mutex is 
unlocked.  

\paragraph{Fluke {\tt mutex} Objects}

The Fluke implementation of mutexes uses Fluke wait queues to keep track
of threads waiting to acquire a particular mutex.

\paragraph{The Promela {\tt Mutex} Type}

Since Promela has builtin atomicity support via the {\tt atomic}
and {\tt d_step} constructs, it was not necessary for us to build our
model Mutexes on top of our model WaitQueues. 
Rather, one bit is sufficient to represent the state of a mutex.
We call Mutexes with one state bit {\it simple} mutexes. They are
implemented in {\tt src/fluke/Mutex-simple.pr}.

However, there are many useful properties that one may wish to verify
about mutexes and how they are used by other code.  For example, it may
be desirable to assert that the same thread that locked a mutex must unlock
it. Also, we might wish to
ensure that the {\it same} thread does not try to lock a mutex it already
holds since this action would result in deadlock.  
We have a second, more heavyweight,
implementation of Promela Mutexes, called {\it safe} mutexes that address
these concerns and aggressively make sanity assertions.  This implementation
can be found in {\tt src/fluke/Mutex-safe.pr}.

Both implementations of Promela Mutexes support the same set of functions.
The intention is that {\tt Mutex-safe.pr} will be used to verify the
appropriate use of Mutexes in a system, and then {\tt Mutex-simple.pr}
will be substituted when the whole system is tested, to minimize state
vector length.


\subsection{Wait Queue Model}

%\xxx{describe wait queues here - how they're implemented,
% what the essential abstraction is, i.e., do we need a guaranteed
% ordering etc., discuss multiprocessor, fairness, starvation}

A wait queue is a data structure used to represent a FIFO list of
threads waiting for a particular event. This is used to keep track 
of the list of threads waiting on a condition variable. 

% We have two implementations of wait queues XXX -- forget it for now --Pat

% -- old paragraph was
% Wait Queues are implemented as a doubly linked list.  The Wait Queue
% itself contains a ``pointer'' to the head and tail.  (The pointer is
% just the id of the thread.)  Additionally, each thread has storage for
% a ``next'' and ``prev'' pointer.  Since it can by definition only ever
% be on one wait queue at a time, this is sufficient.

% new paragraph is
Wait Queues are implemented as a doubly linked list of threads where the 
Thread id are used as pointers. Each thread object contains one ``next''
and one ``prev'' pointer -- this is sufficient since
a thread can only be on one wait queue at a time.
% and goes until here
The interface provided by our model is exactly the same as that
provided by the kernel's C implementation. In addition, the Promela
implementation allows to remove an arbitrary thread from the
wait queue.
 


\subsection{Condition Variable Model}

The C implementation of condition variables
makes use of POSIX-like condition
variables. In addition the interface provides a call which allows a
thread to place another (captured) thread asleep on a condition
variable. Unlike POSIX conditions variables, the associated mutex
is not re-acquired before returning from a wait operation.

A condition variable is modeled using a {\bf wait queue} of the
threads waiting on the condition variable. There is no extra state
required to be maintained as the wait queues implement all the
required features.

\com {
% Not any more, Pat ``fixed'' this.
One change had to be made in the implementation of the condWait()
function because of the way our wait queues were implemented. The
Promela implementation of waitQueueAdd() blocks where as the C
implementation does not. So we had to place some of the statements that
should actually have been executed after the waitQueueAdd(), before it
and place the whole sequence in an atomic block. This leads to a
slight difference in semantics whereby some of the interesting
interleavings are ignored.
}


\subsection{Link Model}

Existing, usable IPC connections can be in two states: active, and
pickled.  Active connections are internally represented as a direct
pointer from on thread to another. (ie, {\tt client\verb\->\ipc_state.server}
is the address of the server's s_thread_t object).  Pickled connections
are represented as {\em links}.  One or the other is accurate, never
both (the non-active link is 0.)  A direct pointer can be
thought of as a 'cached' value of the link.

The kernel provides {\bf links} which represent 
inter-object references.  They are used for
encapsulating pointers and for reference counting.
Initially we will model
these as simply as possible; we may add
verification of the reference counting later.



\subsection{Port and Port Set Models}

Ports are Fluke objects representing targets for IPC requests.
Ports might represent CORBA or MOM objects, for example,
or in Fluke implementations implementing full protection,
they may be used as secure capabilities.
A single server may maintain a number of ports,
each distinguishing a particular object, client,
or object/client pair.

Port sets are objects that act as a rendezvous point
between client threads attempting to perform IPC to a port
and server threads waiting for incoming IPC requests on one or more ports.
Multiple ports can be attached to a single port set,
and multiple server threads can wait for incoming requests
on a single port set at once.
This allows a single ``pool'' of service threads
to service incoming IPC requests from many ports
(i.e. requests associated with many ``objects'' the server exports).

Our model assumes that threads to connect
to for IPC have been ``found'' (the purpose of ports, port sets, and
references). Thus, models of ports or port sets were not
implemented in detail.  The only relevant interfaces used are 
{\tt s_pset_wait_receive()} by servers and {\tt s_port_capture()} by
clients.  The semantics of these operations have been distilled from
the C code.  

% This stuff below has to be made more clear
\com{
Specifically, a thread will block if its ``other half''
hasn't arrived yet.  It will check its cancel pending bit before
blocking.  It will block on a condition variable, in the WAIT_ON_COND
wait state.  When it is awoken it will, by default, return a
KR_RESTART code to its caller.  Additionally, the server thread {\em
always} blocks.  If it arrives and a client is waiting for it, then it
will awaken the client and then block.  In this way it is implied that
the client has the server ``captured'' when it returns from 
{\tt s_port_capture().}}

In our Promela implementation, we model these two functions to implement
the interface and make use of as little state as possible. The
rendezvous point is actually implemented as a bit in the server's
state. If the rendezvous bit is set, it means that either the client
or the server is waiting for a connection.
So when a client wants to connect to a server, it checks the
rendezvous bit to see if the server is waiting for a connection. If
the server is waiting, the server is captured and the connection
is made. If the server is not waiting for a connection, the client
sets the rendezvous 
bit and waits for the server to notice it. When a server is waiting
for an IPC  connection, it checks the rendezvous bit and if a client
is waiting, it makes the connection with the client. Otherwise the
server sets the rendezvous bit and waits for the client. In either
case the server blocks and is captured by the client which eventually
establishes a connection with this server.

% Do we need to say something about the non-cancelability of the wait 
% out here ?

It is interesting to note that the majority of the bugs in our model
came through this interface.  Since we strayed from the actual
implementation of the ports we missed subtle nuances like the server
always being blocked, etc. 



\subsection{Kernel Entry Layer Model}

A thread performing kernel operations is ``controlled'' by the kernel
entry point and the return code. 
All kernel entry points are ``clean.''  A thread that gets canceled
will never reveal itself to be in the middle of a kernel operation;
it will always appear to the canceling thread that it is about to
begin a kernel operation, or that it is in user mode.  To prevent
long or complicated kernel functions from becoming a bottleneck, these
operations are broken into sub-sequences.  

For example the long 
{\tt fluke_ipc_client_connect_send_over_receive()} operation that
connects to a server, sends a request, waits for a reply and receives
it into a buffer, can take quite a while.  As portions of the
operation are completed the kernel entrypoint is advanced.  So, in 
this example, after the ``connect'' phase is completed, the kernel
sets the thread's entrypoint to be {\tt
fluke_ipc_send_over_receive().}

\begin{figure}
{\small
\begin{enumerate}
\item {\tt KR_USER_EXCEPTION} A processor exception occurred which
should be blamed on the user (e.g. because the exception was generated
while accessing user space). The current thread's exception_state
contains the details.

\item {\tt KR_PAGE_FAULT} Page fault occurred.
This gets turned into a real KR_USER_EXCEPTION by the kentry layer
if the page fault cannot be resolved in the kernel
and no appropriate region keeper can be found to handle the fault

\item {\tt KR_CANCEL} Another thread is trying to manipulate us and
has asynchronously canceled us,  e.g. due to thread_interrupt(),
thread_get_state(), thread_set_state().

\item {\tt KR_NO_MEMORY} Ran out of kernel memory.

\item {\tt KR_RESTART} This return code indicates that we have context
switched due to a wait, and we need to restart execution in user mode
before doing anything else in case dependent things have changed.

\end{enumerate}
}
\caption{The set of return codes used within the kernel}
\label{ReturnCodes-fig}
\end{figure}

All of the major functions within the kernel return one of a small,
well defined set of error codes.  See Figure~\ref{ReturnCodes-fig}
for a complete list.
The return code is used to signal special conditions. By convention,
any function returning a non-zero return code signals that the
kernel operation in progress has been canceled or must be restarted.


