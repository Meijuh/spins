\documentstyle[fleqn,epsf,11pt]{article}
\input{amssym.def}
\input{amssym}
\input{vpi.input}


%\pagestyle{headings}
\textwidth 15.5cm
\textheight 21.5cm
\parindent 10pt
\oddsidemargin 0.85cm
\evensidemargin 0.37cm
\def\appendix{%\par
              \setcounter{section}{0}\setcounter{subsection}{0}}
\begin{document}

\title{\sf Verifying Relay Circuits using State Machines}
\author{{\sf P.H.J. van Eijk}\\
       {\footnotesize\sl Utrecht University, Department of Philosophy}\\
       {\footnotesize\sl Heidelberglaan 8, Utrecht, the Netherlands}\\
       {\footnotesize\sl \tt pve@phil.ruu.nl} \\
       {\footnotesize\sl http://www.phil.ruu.nl/home/pve} \\
       {\footnotesize\sl and} \\
       {\footnotesize\sl EDS CVI, PO Box 2233, Utrecht, The Netherlands} \\
       {\footnotesize\sl \tt pve@cvi.ns.nl}\\
       {\footnotesize\sl September 1995}
             }
\date{}
\maketitle

\begin{abstract}
\noindent
######### In this paper we present and discuss a number of techniques that can be used in the modelling
and
verification of electro-mechanical
relay circuits.



\noindent
\footnotesize
{\sl Key words \& Phrases: formal specification, formal verification,
safety critical system, relay, state machines, PROMELA.}
\end{abstract}

\thispagestyle{empty}

\section{Introduction}\label{intro}

For any artefact that is designed, the question sooner or later comes up:
"If we build it this way, will it do what we want it to do?"
This is the correctness question, and the answer to it can be a matter of life and death.
In this paper we will see how current technology for verifying
correctness can be applied
to relatively old hardware technology.
It will become apparent that correctness of relay circuits is a topic with
many sides, and many levels.
On each of these levels there are a number of techniques that can be applied.

This paper is subdivided in the following sections:
\begin{itemize}
\item introduction into electromechanical relay circuits.
\item an example: train mode control.
	\begin{itemize}
	\item combinatorial correctness of the example circuit.
	\end{itemize}
\item PROMELA\cite{GH91}
\item The example studied as a sequential system
\item Verification against state machine models.
\item Conclusion
\end{itemize}
The topic of relay verification has also been discussed by Jacky in \cite{JJ91}.
That work concentrates on {\em combinatorial}
correctness, where the past history of inputs has no relation on the output
of the system and only the current input is relevant.

In this paper we also discuss the correctness of {\em sequential} systems,
whose behaviour depends on the history of inputs or on the interleaving
of events. Such systems exhibit state.
Remarkably, there is little further literature concerning formal approaches to relay
circuits.


\section{Electromechanical relays}

An electromechanical relay, or relay for short, is a hardware device that can implement a
number of digital switching functions.
It typically consists of an electric coil that can exert a magnetic force on a number of contacts.
When electrical power is applied on the coil the contacts either break or make. 
In short, a relay is an electrically operated switch.
Logically, the relay can implement a decoupler or negation function, and the wiring can
implements the logical 'and' and 'or' functions.
In this way circuits can be built that implement arbitrarily complex logical functions.

Relays have been in use for over a hundred years now. Entire telephone switches were once
built out of them.
Despite their relative demise as the result of electronics, and programmable electronics in
particular, there are still applications where their ruggedness and simplicity is vital.
For a number of these applications the same reasons that lead to the deployment of relays make
that the correctness of the circuit is safety-critical. 

A typical design using relays consists of a network of interconnected switches, including the
inputs of the system, that terminate in the relays.
A relay circuit is usually described with a graphical notation that illustrates the
wiring between the components, but not necessarily the physical layout.

There is an almost infinite variety of relay types.
The number of contacts can differ, as well as the
order of there opening and closing. Time-relays exist, that delay their closing for
anything from seconds to  hours, and memory-relays, with two coils, serving as a latch.
\begin{figure}
\epsfxsize=150pt 
\centerline{\epsfbox{figure1.eps}}
\caption{example relay circuit}
\end{figure}



In figure 1 a simple relay circuit is drawn. D is the relay, and A, B and C are contacts or
switches.
The top and bottom lines are wires that power the circuit.
A is a {\em normally-closed} contact, which on activation will interrupt the current. B and C are
{\em normally-open} contacts, which on activation will allow current to flow. The drawing
convention is to show the nonactivated state. In this paper the two cases are also
distinguished by the fact that normally-open contacts are shown to the left, and with a wider
gap.

Logically this circuit can be described as 

\begin{verbatim}
        D = !A && (B || C)
\end{verbatim}

where {\tt !} denotes logical {\tt not}, {\tt \&\&} denotes logical {\tt and},
and {\tt ||} denotes logical {\tt or} (this is PROMELA notation,
 to be explained more fully later).

\section{An example}

Like an automobile, a railway train has several modes of operation.
A car can be off, it can have the accessories on, and it can be fully operational.
These modes are typically controlled through a single key switch.
For a railway train the operational modes determine for example
whether or not the heating and lighting
systems are on, how the doors can be operated, and if the train can actually move. A
typical train can have as much as six major modes.
The system that realises the control over these modes is called the {\em operational mode
control system}.

The passenger train in this example can consist of a fairly arbitrary number of motorcars (each
car has its own engine, like a subway, as opposed to 
the case of a
train that is pulled by a separate locomotive engine),
each of which has two operator cabins, one on each side.
The entire train can be controlled from each cabin, subject to certain requirements.
Some of these requirements are safety-critical. For example, the train must only be able to
move if it is in the 'in service' mode, which can only be entered when an operator key switch
is activated.
The first operator to activate his key must have control, and a second operator (in a
different cabin) must be denied control. Computer scientists will recognize this as a mutual
exclusion problem. 



The operational mode control system (in dutch: bedrijfstoestandskeuze systeem) is typically
implemented using relays, although for newer designs PLC's (Programmable Logic Controller, a
type of microcomputer) are beginning to be applied.
All cabins are equipped with identical copies of a relay circuit. These copies are connected
by a set of wires that run through the entire length of the train.
Together these copies perform the functions of the operational mode control system.
The requirements can be divided in those that apply to each individual copy of the circuit 
and those that apply to the system as a whole.
For example, there may be some kind of exclusion within each individual copy required, as well
as exclusion between all copies.

One of the complications is that the system can consist of a (relatively) arbitrary number of
connected copies of a circuit.
One would want
to validate the system as a whole, independent of the number of connected
copies.


\subsection{The example as a relay circuit}

The circuit that we take as an example here is a simple operational mode control system, it
controls the 'service' state
of a train.
In each cabin a train operator can attempt to set the train in the {\em in service} state through
the use of a key-operated switch.
The function of the circuit is to allow only {\em one} operator to actually control the train,
i.e. the circuit implements a mutual exclusion.
The circuit is drawn in figure 2.

\begin{figure}
\epsfxsize=200pt 
\centerline{\epsfbox{figure2.eps}}
\caption{Two copies of a control circuit}
\end{figure}

The system consists of a number of identical copies of a circuit. Here two copies are drawn.
Each cabin is equipped with one such circuit.
The top and bottom horizontal lines are the electrical power feed, and the middle horizontal
line (labelled 'InService') is the one that connects all cabins together.
This line signals whether or not the entire
train is in service, it also functions as a distributed OR. Presumably, other train systems, such as the door and traction systems,
are controlled on the basis of this signal.

The relation between the contacts and their relays is indicated by the name that they are
labelled by.
We
will append a digit to the names to indicate the copy of the circuit that they are in, e.g. {\tt servicelocal_1}.
All contacts are shown in the state in which the relays are not activated.

\subsection{Combinatorial correctness}

The combinatorial behaviour of the circuit can be formalised with the following equations:

\begin{verbatim}
        servicelocal_1 = key_1 && !serviceremote_1
        serviceremote_1 = inservice && (!key_1 || serviceremote_1)

        servicelocal_2 = key_2 && !serviceremote_2
        serviceremote_2 = inservice && (!key_2 || serviceremote_2)

        inservice = servicelocal_1 || servicelocal_2
\end{verbatim}

For any input, all combinations of values that fulfill these equations represent a stable state of the
circuit.
Conversely, if the equations are not satisfiable for all inputs the circuit is not stable.
An example of a system that is not stable is simply:
\begin{verbatim}
        k = !k
\end{verbatim}
A relay implementation of this will most likely exhibit an oscillation.



The correctness criteria on the circuit are:

\begin{verbatim}
        !(servicelocal_1 && servicelocal_2)
\end{verbatim}
meaning that only one of the cabins can be the operating cabin, and

\begin{verbatim}
        !(servicelocal_1 && serviceremote_1) &&
        !(servicelocal_2 && serviceremote_2)
\end{verbatim}
which is a type of consistency check on the settings of the relays of one cabin. The relays should not
simultaneously indicate that this is the operating cabin, and that some other cabin is the
operating cabin.

The behaviour equations, taken as a logical proposition, should logically imply these
correctness criteria.
Actually, the first one is not true. There is a race condition if two keys are switched
simultaneously..


\section{PROMELA}

PROMELA\cite{GH91} is a language designed for the modelling of communicating finite state
machines. It was designed for the modelling of communication protocols, but it can be used for
the analysis of almost any parallel or distributed computer system.
Beyond these applications, it can even be used to model systems (such as relay circuits)
that are not traditionally
seen as computing devices.
In this paper we show the various levels at which PROMELA can be used to analyse relay
circuits.

PROMELA as a language is complemented by SPIN, a software package for the analysis of
specifications written in PROMELA. SPIN can be used in a number of ways. In an interactive
mode it can provide a simulation of the behaviour of a system. In an automatic mode it can try
to exhaustively analyse the statespace of a system, proving or refuting user specified
correctness requirements.

(hier nog wat beschrijving van features uit Cattel's paper overnemen)


\section{Modelling Sequential behaviour}


In the above we have looked at the modelling of relay circuits by simple combinatorial equations.
At that level there are a number of correctness criteria that
can be expressed, and verified or refuted.

In this view relays are in one of two states.
Yet, the story is not that simple. All switching elements have delays, and these delays can
influence the correctness of the system.
What happens in fact is that  each switching activity progresses through a sequence of
 at least three phases or states.
Suppose the
relay is unpowered, or off. That phase ends when the power is applied, and here an
intermediate phase starts. From this point on the
magnetic force starts to build up, and above a certain threshold the contacts start to move
from their resting positions.
A 'break' contact will break as soon as it has started to move,
a 'make' contact will make as soon as it has reached its end position (hopefully it will not
bounce back from there).
When all contacts have done so, the intermediate phase is over, and the last phase begins: the realy is truly activated.
There is not necessarily a fixed order in which the contacts make or break, although some
types of relays have been designed to exhibit an explicit order, e.g. make before break or
vice versa.

The fact that relays have this type of
behaviour can have an impact on the correctness of the circuit.
There can be situations in which there is literally a race between two contacts to reach their
positions, and the behaviour of the system depends on the winner of the race.
In cases like these it is thus important in an analysis to consider all possible sequences of operations.

For some types of relays this behaviour is also influenced by aging.
The implication of that is that such systems can fail after a long time of correct operation.

\subsection{Modelling Approach}

How to model this intermediate stage? 
There is a parallel between the delay in the closing of a relay contact, and the delay
a message can experience on its way from sender to receiver.
It is tempting to use this analogy in modelling a relay circuit in PROMELA. A relay is then a
process with an incoming message (the electrical current) and a number of outbound messages, one for each
contact. The wires would then be the message conduits or channels.
Appealing in this approach is that it is truly event based, and that all possible
interleavings of events will be considered by the analyser.
Unfortunately this approach has its flaws.
It is awkward to model junctions of wires in this
way. They will become processes with multiple inputs and/or outputs.
Some wires are 
bidirectional (such as InService in the example), which also adds to the complexity.

Instead of looking at events, we can also look at states.
The voltage on a wire is a state, represented by a variable,
and (re)computing the position of the relay contacts is
then a state change, which can be
represented by an assignment.
Each relay now corresponds to one assignment statement, with the left hand
side variable representing the coil, and the right hand side expression
representing the combinatorial logic of wires and contacts.
Intermediate wires (in the example {\tt inservice}) can also be represented
explicitly with an assignment statement.

The behaviour of the system can now be described by a set of assignment statements that are
randomly executed.
All reachable states are covered by this model.

The system can be described in PROMELA as follows. Every wire could be described by a separate
variable, but for reasons of simplicity only a minimal subset is taken. All variables are
global, so as to be able to inspect and modify them, where appropriate.

\begin{verbatim}
proctype tks()
{
do
::	servicelocal_1 = key_1 && !serviceremote_1; skip
::	serviceremote_1 = inservice && (!key_1 || serviceremote_1); skip

::	servicelocal_2 = key_2 && !serviceremote_2; skip
::	serviceremote_2 = inservice && (!key_2 || serviceremote_2); skip

::	inservice = servicelocal_1 || servicelocal_2; skip
od
}

proctype monkey()
{
do
::	key_1=1; skip
::	key_1=0; skip

::	key_2=1; skip
::	key_2=0; skip

od
}

init
{       run tks();
        run monkey();
}
\end{verbatim}

The {\tt monkey} process\footnote{Named after the saying: {\em if you give enough monkeys
enough typewriters, you will find a Shakespeare sooner or later.}}
can arbitrarily turn all key switches on and off.
Two parallel processes, one representing the system, and the other representing the
environment using the system, together describe an entire state space. The occurrence of {\tt
skip} instructions is for technical reasons.
Without them, the process {\tt tks} would have only one state, which would make the output of
SPIN less clear.
On models along these lines SPIN will now be able to explore all behaviour
that is electromechanically possible.
For example, make before break or vice-versa can be studied, and even relay deavtivation before all contacts are made.

The requirements on the model are the same as before, and can be described in PROMELA as follows.

\begin{verbatim}
proctype checks()
{
   assert(!(servicelocal_1 && servicelocal_2));
   assert(!(servicelocal_1 && serviceremote_1))
}
\end{verbatim}

Not surprisingly, the same race condition still persists in the system. If the two switches are
turned on together, there will be no mutual exclusion
(For the system to be really correct, this circuit would need to implement a distributed leader election\cite{LE} function).

The assumption under which the circuit will work correctly, is that no external input will be
given as long as the circuit is still processing. In other words, the circuit has to be
{\em stable} or {\em quiescent} before an additional input is given.
In the model this means that every assignment that can have an effect must have been done.
This is expressed by translating every assignment of the form {\tt x = E} into a condition
{\tt x == E}, and taking the conjunction of all these conditions.
In the example this boils down to adding the following PROMELA code.

\begin{verbatim}
proctype tks()
{
do
/* as before */
::      stable=
        servicelocal_1 == (key_1 && !serviceremote_1)) &&
        (serviceremote_1 == (inservice && (!key_1 || serviceremote_1))) &&
        (servicelocal_2 == (key_2 && !serviceremote_2)) &&
        (serviceremote_2 == (inservice && (!key_2 || serviceremote_2))) &&
        (inservice == (servicelocal_1 || servicelocal_2));  stable=0
od
}
\end{verbatim}

Of all the states of this process, there is only one state where another process can assume that
the system is stable, and that is the state at the last semicolon.
If the variable {\tt stable} is true, the model will be in this state.

Expressing that the system will only be operated in a stable state
can be done with the following monkey process.

\begin{verbatim}
proctype monkey()
{
do
::	stable; key_1=1; stable=0
::	stable; key_1=0; stable=0

::	stable; key_2=1; stable=0
::	stable; key_2=0; stable=0

od
}
\end{verbatim}

The keys can only be operated when the system is stable,
 and they can be operated only once at a time.
The correctness requirements in this model need to 'lock-out' the system
process, which is done here with an 'atomic' statement.
Note that the monkey process does not need to be locked-out, as the system
will not react to it during the assertion.

\begin{verbatim}
proctype checks()
{ atomic{
   stable;
   assert(!(servicelocal_1 && servicelocal_2));
   assert(!(servicelocal_1 && serviceremote_1))
	}
}
\end{verbatim}

Additional embellishments and refinements can be added to this model.
This model assumes that all relays are indeed in their nonactivated state
when the system is initialised.
This is realistic for normal relays, but not so for memory relays or external switches.
It is a small matter to add a nondeterministic assignment of values to variables in the initialisation phase. This would then expose all errors
that may be the result of an incorrect initialisation.

In case the correctness of a circuit critically depends on the fact that
a certain contact makes before another breaks, this can be expressed by putting the corresponding assignments in sequence rather than in parallel.

\section{Relay circuits as implementations of state machines}


On a certain level of abstraction the circuit is a state machine, with external stimuli as
true events. Some relay circuits are required to actually remember something, rather than just
compute some logical function.
It can be convenient to describe the actual state machine, and check it against the relay circuit.
On the state machine itself certain validations are also possible: can every state be
reached?


Pushing the button so fast that the relays can't handle it.

You will also see behaviour that will occur if only a subset of the contacts will have been
operated, (is toch ook al bij de vorige)
or not at all.

Again discussion on stable.

\section{Concluding Remarks and Future Research}

Relay systems can be analysed in great detail, right down to the level of 'meaningless'
errors.


\begin{thebibliography}{10}


\bibitem{LE}
D.~Dolev, M.~Klawe, and M.~Rodeh.
\newblock An {$O(\mbox{$n$ log $n$})$} unidirectional distributed
algorithm for
  extrema finding in a circle.
\newblock {\em Journal of Algorithms}, (3):245--260, 1982.

\bibitem{HK94}
J.F.~Groote, J.W.C.~Koorn and S.F.M.~van Vlijmen.
\newblock The sfaety guaranteeing system at station {Hoorn-Kersenboogerd}.
\newblock Technical Report 121, Logic Group Preprint Series, Utrecht University, 1994.

%\bibitem{BKOV94}
%R.N. Bol, J.W.C. Koorn, L.H. Oei, and S.F.M. van Vlijmen.
%\newblock Syntax and static semantics of the interlocking design 
%and application language.
%\newblock Report P9422, Programming Research Group, University of Amsterdam,
%1994. To appear.

\bibitem{JJ91}
J.~Jacky
\newblock Verification, Analysis and Synthesis of Safety Interlocks.
\newblock Technical Report 91-04-01, Radiation Oncology Department, University of Washington,
1991

\bibitem{GH91}
G.~Holzmann.
Design and Validation of Computer Protocols.
Prentice Hall, New Jersey, 1991, ISBN 0-13-539925-4. 
\\
{\tt http://netlib.att.com/netlib/att/cs/home/holzmann-spin.html}
\end{thebibliography}

\end{document}

