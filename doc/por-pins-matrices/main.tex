\documentclass{paper}

\usepackage{multirow} 
\usepackage{graphicx}
\usepackage{cite}
\usepackage{comment}
\usepackage{color}
\usepackage[lined,vlined,linesnumbered,boxed]{algorithm2e}
\usepackage[pdftex,
        colorlinks=false,
        pdftitle={Scalable Recursive State Compression},
        pdfauthor={Alfons Laarman},
        pdfsubject={},
        pdfkeywords={compression, model checking, tree-based algorithm},
        %pdfadjustspacing=1,
        pagebackref=true,
        pdfpagemode=None,
        bookmarksopen=true]{hyperref}
        
\usepackage{ntabbing, float}
\usepackage[cmex10]{amsmath}
\usepackage{amssymb}
\usepackage{amsfonts}
\floatstyle{boxed}
\newfloat{fig}{bt}{fig}
\floatname{fig}{Figure}

\interfootnotelinepenalty=10000
\renewcommand{\bottomfraction}{1.2}
\renewcommand{\topfraction}{1.2}

\title{Guards-based Partial Order Reduction}
\subtitle{}

%\author{Alfons Laarman\label{}\\
%	a.w.laarman@ewi.utwente.nl
%}

\newtheorem{definition}{Definition}
%\newcommand{\l}{\ \|\ }

\pagestyle{empty} 
\pagenumbering{arabic} 
\begin{document}

\maketitle
\thispagestyle{empty} 

\section{Guard Formulas}

Thus far, we have discussed guards only as interpretations over states 
(evaluating to true or false). In order to demonstrate how
guard-based partial-order reduction can be used generically and 
efficiently, we will now discuss the form of guards.
We show how different modeling languages can compile to guard formulas, 
and how to calculate the different dependency relations needed for 
partial-order reduction. In fact, we show that - even with our generic 
definition - we can calculate the needed relations in more detail than
with the language-specific heuristics usually discussed in literature.

\begin{definition}
The syntax of a guard formula $\gamma$ is defined as follows:\label{}\\	
$\gamma ::= \gamma\lor\gamma | \gamma\land\gamma | sp$
\end{definition}

Here, $sp$ is a simple predicate over atoms $a$ of the following form:\label{}\\
$sp ::= sp=sp | sp\neq sp | sp<sp | sp > sp | sp\le sp | sp\ge sp | a$

An atom can be a constant (number) or an identifier (variable).


\subsection{Language Modules}

\subsection{Generic Calculation of Dependency Relations}
\newpage

\newcommand\promela{\textsc{Promela}\xspace}
\newcommand\pins{\textsc{Pins}\xspace}

\newcommand\guards{\ensuremath{\mathcal{G}}\xspace}
\newcommand\actions{\ensuremath{\mathcal{A}}\xspace}
\newcommand\trans{\ensuremath{\mathcal{T}}\xspace}
\newcommand\bool{\ensuremath{\mathbb{B}}\xspace}
\newcommand\writeset{\ensuremath{\mathcal{W}}\xspace}
\newcommand\readset{\ensuremath{\mathcal{R}}\xspace}
\newcommand\MC{\ensuremath{\mathcal{MC}}\xspace}
\newcommand\NES{\ensuremath{\mathcal{NES}}\xspace}
\newcommand\NDS{\ensuremath{\mathcal{NDS}}\xspace}

\newcommand\vars{\ensuremath{\mathcal{V}}\xspace}
\newcommand\visible{\ensuremath{\mathit{vis}}\xspace}
\newcommand\nguards{\overline{\guards}\xspace}

\newcommand\nes{\textsf{nes}\xspace}
\newcommand\nds{\textsf{nds}\xspace}
\newcommand\enables{\textsf{enables}\xspace}
\newcommand\agrees{\textsf{agrees}\xspace}
\newcommand\mce{\textsf{mce}\xspace}
\newcommand\conflicts{\textsf{conflicts}\xspace}
\newcommand\true{\ensuremath{\mathit{true}}\xspace}
\newcommand\false{\ensuremath{\mathit{false}}\xspace}

\newcommand\tie{\bowtie\hspace{1ex}}

%\begin{minipage}{\linewidth}

$\equiv$ is used for syntactical equivalence check;
for both \enables and \mce, we can assume the same state).

$c_1, c_2$ are constants.

$\bowtie$ a comparison operator (in a simple linear (sub)equation
of a guard).

$v,v'$ are state variables.

$\nguards = \{ \neg g \mid g\in\guards\}$ are the set of inverses of all
guards.


%\begin{subequations}\begin{align} 
%\begin{array}{l@{}l@{}r@{}l}
%\mce(g_1\lor g_2,g') &:= \mce(&g_1adshjdasjka,&g')\lor\mce(g_2, g')\\
%\mce(g_1\lor g_2,g')kldfnfklds &:= \mce(&g_1,&g')\lor\mce(g_2, g')\\
%\end{array}
%\end{align}\end{subequations}


\newcommand*{\Comment}[1]{\hfill\makebox[3cm][r]{\textit{$\triangleright$#1}}}%

\begin{algorithm}[p]
\begin{ntabbing}\hspace{-.8ex}
$\conflicts\colon \guards \times \guards \rightarrow \bool$\label{}\\
$\conflicts(\dots, \true)$ \=:= \false\label{}\\
$\conflicts(\true, \dots)$ \=:= \false\label{}\\
$\conflicts(v \bowtie c_1, v'\bowtie c_2)$ \=:= $v\equiv v'\land$
\textit{check linear equation}
\label{}\\
$\conflicts(\dots, \dots)$ \>:= $\true$ \Comment{unknown}
\label{}\\
\end{ntabbing}\vspace{-2em}
\caption{(Simple) guard conflicts}
\end{algorithm}
\vspace{-3em}

\begin{algorithm}[p]
\begin{ntabbing}\hspace{-.8ex}
$\mce\colon\guards\times\guards\rightarrow \bool$\=~~~~~~~~~~~\=~~~~~~~~~~~~~~~~~~~~~\=
\label{}\\
$\mce(g_1\lor g_2,$\>$g')$ \>:= $\mce(g_1,$\>$g')\lor\mce(g_2, g')$\label{}\\
$\mce(g_1\land g_2,$\>$g')$ \>:= $\mce(g_1,$\>$g')\land\mce(g_2,g')$\label{}\\
$\mce(\neg(g_1\lor g_2),$\>$g')$\>:= $\mce(\neg g_1,$\>$g')\land\mce(\neg g_2,g')$
\label{}\\
$\mce(\neg(g_1\land g_2),$\>$g')$\>:= $\mce(\neg g_1,$\>$g')\lor\mce(\neg g_2,g')$
\label{}\\
$\mce(\neg(v~\bowtie~g),$\>$g')$\>:= $\mce(v~\overline{\bowtie}~g,$\>$g')$
\label{}\\
$\dots$\textit{~by symmetry:~$g, g'\Rightarrow g', g$}\label{}\\

$\mce(g, $\>$g')$ \>:= $\conflicts(g,g')$
\label{}\\
\end{ntabbing}\vspace{-2em}
\caption{Maybe coenabled computation}
\end{algorithm}
\vspace{-3em}

% \accentedsymbol{\dbarA}{\Bar{\Bar{A}}}


\begin{algorithm}[p]
\begin{ntabbing}\hspace{-.8ex}
$\conflicts\colon \actions \times \guards \rightarrow$
\=$\bool$~~~~~~~~~~~\=
\label{}\\

$\conflicts(v:=c_1,$\>$v'\bowtie c_2)$\>:=
$\conflicts(v=c_1,v'~\bowtie~c_2)$
\Comment{90\% of all actions}\label{}\\

$\conflicts(v\texttt{++},$\>$v'\bowtie\dots)$\>:=
$v\equiv v'\land\tie\in\{<,\leq\}$
\label{}\\
$\conflicts(v\texttt{--}, $\>$v' \bowtie \dots)$ \>:=
$v\equiv v'\land\tie\in\{>,\geq\}$
\label{}\\
$\conflicts(\alpha, $\>$\neg (v \bowtie g))$ \>:=
$\conflicts(\alpha,v~\overline{\bowtie}~g)$
\label{}\\
$\conflicts(\dots, $\>$\dots)$ \>:= $\true$ \textit{// unknown}
\label{}\\
\end{ntabbing}\vspace{-2em}
\caption{Action / (simple) guard conflicts (note the $\equiv$)}
\end{algorithm}
\vspace{-3em}


\begin{algorithm}[p]
%\texttt{
\begin{ntabbing}\hspace{-.8ex}
$\MC_t$ := $\{ (t,g)\in\trans\times\guards\cup\nguards \mid
\forall g'\in\guards(t)$\=$\colon \mce(g,g') \}$
\label{}\\
\\
$\nes,\nds\colon \guards \rightarrow 2^\trans$\=\label{}\\
$\nes(g)$ \>:= $\{ t\in \trans \mid (t, \neg g)\in\MC_t
\land \enables(t, g) \}$
\label{}\\
$\nds(g)$ \>:= $\nes(\neg g) $
\label{}\\
\\
$\enables\colon\trans \times \guards \rightarrow \bool$~~~~~~\=
\label{}\\
$\enables(t,g_1\land g_2)$\>:= $\enables(t,g_1)\lor\enables(t,g_2)$
\label{}\\
$\enables(t,g_1\lor g_2)$\>:= $\enables(t,g_1)\lor\enables(t,g_2)$
\label{}\\
$\enables(t,\neg(g_1\lor g_2))$\>:=
$\enables(t,\neg g_1)\lor\enables(t, \neg g_2)$
\label{}\\
$\enables(t,\neg(g_1\lor g_2))$\>:=
$\enables(t,\neg g_1)\lor\enables(t, \neg g_2)$
\label{}\\
$\enables(t, g)$\>:= $\agrees(t, g)$
\label{}\\
\\
$\agrees\colon \trans \times \guards \rightarrow \bool$
\label{}\\
$\agrees(t, g)$ := $\exists \alpha\in\actions(t)\colon
\writeset(a)\cap\readset(g)\neq\emptyset \land \conflicts(\alpha,\neg g)$
\label{}\\
\end{ntabbing}\vspace{-2em}
\caption{NES/NDS computation (extend guard set with negated gaurds
$\nguards$) (TODO: search for smallest sets)}
\end{algorithm}

\newpage 


To reduce the size of the computed \nes and obtain better reductions,
it is tempting to
use the maybe coenabled relation at the leafs of
the \agrees function, where we then check whether $t$ is already
be coenabled with $\neg g$. However, such an attempt would be incorrect,
as the following example illustrates. When we call
$\nes(t,g)$ with $g \equiv g_1 \land g_2$ and $g_1 \equiv pc_0 \neq 5$,
it eventually checks whether $g_1$
can be enabled by an action $\alpha\in\actions(t)$.
If there exists an $\alpha \equiv pc_0 := 2$, it is considered enabling. 
Now \agrees can additionally check whether
$(\neg g_1, t) \not\in \MC$, with e.g. $\guards(t) = \{pc_0=1\}$,
and conclude that an already (co)enabled
transition is not \emph{necessarily} enabling.
However, because $g$ is a conjunction,
we may still have $(\neg g, t)\in\MC$.

Whether or not we may mark $t$ as not necessary enabling for $g$ depends
then on $g_2$. If $g_2 \equiv pc_1 = 1$, then indeed we may. If, however,
$g_2 \equiv pc_0 = 1$ this is no longer the case.
But what if $g_2\equiv pc_1 = 10$?
This depends again on the actions set of $t$. If $\actions(t)$
contains an action $\beta\equiv pc_1 := 10$ then it should be.
But if no action of $t$ writes to $pc_1$, then why would it be?
A solution seems to be to limit the \mce check to those subexpressions of
guards whose test set intersects the write set of the transition.


\begin{enumerate}
\item Computation \mce + dependent for
	$(\guards\times\trans)\times(\guards\times\trans)$ is more efficient.
\item 
\end{enumerate}

%\end{minipage}

\newpage

For LTL model checking, we the visibility proviso needs to be implemented.
It states that a partial persistent set ($en(s) \neq \mathit{ample(s)}$)
does not contain any visible transition, where the definition of 
visibility is:
\begin{quote}
A transition that may enable or disable the LTL formula.
\end{quote}
In practice, this is over estimated by selecting those transitions that
influence the closure of the LTL formula, i.e. its smallest simple
sub formulas.

An LTL formula $F$ can be expressed with references to state variables
$v \in \vars(F)$, e.g. $\Box \Diamond v = 5$.
In this case, the visibility proviso can be implemented by marking
all transitions that write to these variables as visible:
\[
\trans^F_\visible:=\{t\in\trans\mid\writeset(t)\cap\vars(F)\neq\emptyset\}
\] 

However, this is a coarse over approximation that can easily lead to
significantly less reduction. Therefore, we may just as well employ guards
and other state labels to define our LTL properties.
For instance, $\Box \Diamond \mathit{np\_}$ can express the
livelock property, where, in \promela, $\mathit{np\_}$ indicates that
no process counter is at statement labeled with \textsf{progress:}.
We extend the definition of \nes, \nds and \mce with all state labels
(including guards), so that we can now
define a more fine-grained  set of visible transitions:

\[
\trans^F_\visible:=
\{ t\in\trans\mid\writeset(t)\cap\vars(F)\neq\emptyset \}
\cup
\bigcup_{g\in\guards(F)} \nes(g)\cup\nds(g)
\]

This computation can completely be done in the front end.

\section{Conclusion}

Summarizing: to enable full partial order reduction including LTL
we only require the front-end to compute and export:
\begin{enumerate}
\item Transition-state dependencies, as in the 
original \pins: \writeset and \readset,
\item guards as \pins state labels and their relation to transitions: 
	\guards, $\guards(\trans)$, and
\item a guard-level maybe coenabled matrix:
	$\MC := \{ g,g\in\guards \mid  \mce(g,g)\}$.
\end{enumerate}

To improve reduction the frontend can also offer a fine-grained necessary 
enabling set, which is normally over estimated using the dependencies:
\begin{enumerate}
\item A necessary enabling set:
	$\NES := \{ (g,t)\in\guards\times\trans \mid t\in\nes(g)\}$.
\end{enumerate}

This also enables LTL model checking, as the visibility proviso can also
be roughly over-approximated using the dependencies. To improve the
reductions under LTL, we additionally require:
\begin{itemize}
\item Negated guards in all dependencies: $\guards':=\guards\cup\nguards$,
	or
\item a necessary disabling set: 
	$\NDS := \{ (g,t)\in\guards\times\trans \mid t\in\nes(\neg g)\}$.
\end{itemize}
Because $\nds(g) = \nes(\neg g)$, the first option hardly increases the
amount of effort required to implement.

\bibliography{main}
\bibliographystyle{plain}

\end{document}
















